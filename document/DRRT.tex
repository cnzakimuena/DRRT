\documentclass{article}
\usepackage[utf8]{inputenc}
\usepackage{amsmath}

\title{Deep Recurrent Relation Transformer}
\author{Belanger Nzakimuena, C. M.}
\date{January 2022}

\begin{document}

\maketitle

\noindent A mathematical description of the Deep Recurrent Relation Transformer (DRRT) is provided.  DRRT consists of a Deep Relation Transformer (DRT) as implemented by Song et al (2021), enhanced with local recurrence.

\section{Feature Extraction}

\noindent $A = \left\{a_{1}, a_{2}, a_{3}, ..., a_{n}\right\}$ is the features corresponding to the first input $x_{1}$. \\
\noindent $B = \left\{b_{1}, b_{2}, b_{3}, ..., b_{m}\right\}$ is the features corresponding to the second input $x_{2}$. \\

\noindent Feature extraction stage local recurrence feed-forward equations for $A'$ are given by,

\begin{align}
h_{t}^{A} &= \sigma_{h}(i_{t}) = \sigma_{h}(U_{h}A_{t}+V_{h}h_{t-1}^{A}+b_{h}) \\
A'_{t} &= \sigma_{A'}(j_{t}) = \sigma_{A'}(W_{A'}h_{t}^{A}+b_{h})
\end{align}

where \\

\indent\indent $A_{t}$ is the local RNN input vector \\
\indent\indent $h_{t}^{A}$ is the hidden layer output vector \\
\indent\indent $A'_{t}$ is the local RNN output vector \\
\indent\indent $b_{h}$ is the bias vector \\
\indent\indent $U, W$ are parameter matrices \\
\indent\indent $V$ is a parameter matrice \\
\indent\indent $\sigma_{h}, \sigma_{A'}$ are activation functions \\

\noindent Feature extraction stage local recurrence feed-forward ouput $B'$ is calculated in a similar way.

\section{Relation Modules}
\subsection{Global Relation Module}

\noindent Global relation maps (GRM) for second input features $\alpha^{g}$ and for first input features $\beta^{g}$ are calculated by,

\begin{align}
\alpha^{g} &= Softmax\left(\frac{\Psi^{g}(A', B')}{\sqrt{d_{B'}}}\right) \\
\beta^{g} &= Softmax\left(\frac{\Psi^{g}(A', B')}{\sqrt{d_{A'}}}\right)
\end{align}

where \\

\indent\indent $d_{B'}$ represents the dimension of second input features \\
\indent\indent $d_{A'}$ represents the dimension of first input features \\

\noindent The global pairwise function $\Psi^{g}$ is calculated by,

\begin{align}
\Psi_{ij}^{g}(A', B') &= W_{g}[W_{a'}a'_{i}; W_{b'}b'_{j}] 
\end{align}

where \\

\indent\indent $a'_{i}$ is the $i$-th first input region, $b'_{j}$ is the $j$-th second input region \\ 
\indent\indent $W_{g}$, $W_{a'}$ and $W_{b'}$ are weight matrices to be learned \\ 
\indent\indent $[W_{a'}a'_{i}; W_{b'}b'_{j}]$ denotes the concatenation of $W_{a'}a'_{i}$ and $W_{b'}b'_{j}$ \\

\subsection{Guided Regional Relation Module}

\noindent First and second input features are divided into $K$ regions, \\

\indent\indent $A'^{R} = \left\{A_{1}^{'R}, A_{2}^{'R}, A_{3}^{'R}, ..., A_{K}^{'R}\right\}$ \\ 
\indent\indent $B'^{R} = \left\{B_{1}^{'R}, B_{2}^{'R}, B_{3}^{'R}, ..., B_{K}^{'R}\right\}$ \\ 

\noindent $B_{k}^{'R}$ and $A_{k}^{'R}$ denote second and first input features in $k$-th corresponding region, \\

where \\

\indent\indent $k \in \left\{1, 2, 3, ..., K\right\}$ is the region index \\

\noindent Regional relation maps for second input features in $k$-th region $\alpha^{r(k)}$ is defined as,

\begin{align}
\alpha^{r(k)} &= \frac{1}{C_{r}}Softmax\left(\Psi^{r(k)}(A', B')\right)
\end{align}

\noindent Where $\Psi^{r(k)}(A', B')$ projects the topographic correspondence $B'-A'$ feature vector to a scalar expressed as,

\begin{align}
\Psi_{ij}^{r(k)} &= R_{ij}^{r(k)}W_{r}[W_{a'}a'_{i}; W_{b'}b'_{j}] 
\end{align}

with

\begin{align}
R_{ij}^{r(k)} &= \begin{cases} % \, denotes spacing
		1, a'_{i} \in A_{k}^{'R}\: and\: b'_{j} \in B_{k}^{'R} \\ 
		0, otherwise
	\end{cases} 
\end{align}

\noindent Where region pair factor $R_{ij}^{r(k)}$ is $1$ when $a'_{i}$ and $b'_{j}$ are in the $k$-th topographic correspondence region pair or $0$ when they are not in the $k$-th topographic correspondence region. \\
\noindent $W_{r}$, $W_{a'}$ and $W_{b'}$ are learnable weight vectors which are weight sharing for computing $K$ regional relations. \\

\noindent The normalization factor is given by,

\begin{align}
C_{r} &= \frac{\sum_{i, j, k}\left(R_{ij}^{r(k)}\right)}{\sqrt{d_{B'}}}
\end{align}

\noindent Regional relation maps for first input features in $k$-th region $\beta^{r(k)}$ is calculated in a similar way.

\section{Interaction Transformer Module (RNN enhanced)}
 
\noindent Relation modules stage local recurrence feed-forward equations for $\alpha^{'g}$ and $\beta^{'g}$, and $\alpha^{'r(k)}$ and $\beta^{'r(k)}$ are calculated in a similar way as outputs at the feature extraction stage. \\

\noindent The information transformation process can be formulated as,

\begin{align}
T_{b'\rightarrow a'_{i}}^{\sigma} &= \sum_{\forall j}\alpha_{ij}^{'\sigma}W_{\tilde{b'}}b'_{j} \\
T_{a'\rightarrow b'_{j}}^{\sigma} &= \sum_{\forall i}\beta_{ij}^{'\sigma}W_{\tilde{a'}}a'_{i}
\end{align}

where \\

\indent\indent $\sigma \in {g, r(k)}$ \\
\indent\indent $W_{\tilde{b'}}$ and $W_{\tilde{a'}}$ are weight matrices to be learned \\

The outputs of the information transformation process is combined as,

\begin{align}
A'_{b'\rightarrow a'} &= Fusion\left(T_{b'\rightarrow a'}^{g}, \sum_{k}T_{b'\rightarrow a'}^{r(k)}\right)W_{t} \\
B'_{a'\rightarrow b'} &= Fusion\left(T_{a'\rightarrow b'}^{g}, \sum_{k}T_{a'\rightarrow b'}^{r(k)}\right)W_{t}
\end{align}

where \\

\indent\indent $T_{b'\rightarrow a'}^{g}$ is computed based on the global relation map \\
\indent\indent $T_{b'\rightarrow a'}^{r(k)}$ is computed based on the $k$-th regional relation map \\
\indent\indent $W_{t}$ is the parameter matrices to be learned \\
\indent\indent $W_{t}$ in Eq. $12$ and Eq. $13$ are weight sharing \\
\indent\indent $A'_{b'\rightarrow a'}$ is the transformed first input features \\
\indent\indent $B'_{a'\rightarrow b'}$ is the transformed second input features \\

\noindent Three different fusion methods (sum, maxout or concatenation) can be used to combine global and regional representations. \\

\noindent The original and transformed features are combined as the output of the interaction transformer module as,

\begin{align}
Z_{ab} &= A' + A'_{b'\rightarrow a'} \\
Z_{ba} &= B' + B'_{a'\rightarrow b'}
\end{align}

\noindent Second input ($Z_{ba}$) and first input ($Z_{ab}$) representations are combined by average fusion then passed to a three-layer fully connected classifier to generate a classification.

% \begin{equation}
% \begin{aligned}
% \frac{1}{N}\sum_{k=0}^{N-1}X_{k}E_{N,k} &= \mathcal{L}^{-1}\left\{f_1(\delta).f_2(\delta)\right\} \\
% \frac{1}{N}\sum_{k=0}^{N-1}X_{k}E_{N,k} &= \exp(mt) \star \left\{\frac{l}{2\sqrt{\pi t^3}} exp(-l^2/{4t})\right\} \\
% & = F_1 * F_2
% \end{aligned}
% \end{equation}

\end{document}